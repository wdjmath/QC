\chapter*{Preface}

With availability of near-term quantum devices and the breakthrough of quantum supremacy experiments, quantum computation has received an increasing amount of attention from a diverse range of scientific disciplines in the past few years. 
Despite the availability of excellent textbooks as well as lecture notes such as \cite{NielsenChuang2000,KitaevShenVyalyi2002,Nakahara2008,RieffelPolak2011,Aaronson2013,PreskillQuantumLec,DeWolfQuantumLec,ChildsQuantumLec}, these materials often cover \textit{all} aspects of quantum computation, including complexity theory, physical implementations of quantum devices, quantum information theory, quantum error correction, quantum algorithms etc. This leaves little room for introducing how a quantum computer is supposed to \textit{be used} to solve challenging computational problems in scientific and engineering. 
For instance, after the initial reading of (admittedly, selected chapters of) the classic textbook by Nielsen and Chuang~\cite{NielsenChuang2000}, I was both amazed by the potential power of a quantum computer, and baffled by its practical range of applicability: are we really trying to build a quantum computer,  either to perform a quantum Fourier transform or to perform a quantum search? 
Is quantum phase estimation the only bridge connecting a quantum computer on one side, and virtually \textit{all} scientific computing problems on the other, such as solving linear systems, eigenvalue problems, least squares problems, differential equations, numerical optimization etc.?

Thanks to the significant progresses in the development of quantum algorithms, it should be by now self-evident that the answer to both questions above is \textit{no}. 
This is a fast evolving field, and many important progresses have only been developed in the past few years. 
However, many such developments are theoretically and technically involved, and can be difficult to penetrate for someone with only basic knowledge of quantum computing.
I think it is worth delivering some of these exciting results, in a somewhat more accessible way, to a broader community interested in using future fault-tolerant quantum computers to solve scientific problems. 

This is a set of lecture notes used in a graduate topic class in applied
mathematics called ``Quantum Algorithms for Scientific Computation'' at the Department of Mathematics, UC Berkeley during the fall semester of 2021. 
These lecture  notes focus only on quantum algorithms closely related to scientific computation, and in particular, matrix computation.  In fact, this is only a small class of quantum algorithms viewed from the perspective of the ``quantum algorithm zoo''\footnote{\url{https://quantumalgorithmzoo.org/}}. This means that many important materials are consciously left out, such as
quantum complexity theory,  
applications in number theory and cryptography (notably, Shor's algorithm), applications in algebraic problems (such as the hidden subgroup problems) etc. 
Readers interested in these topics can consult some of the excellent aforementioned textbooks. 
Since the materials were designed to fit into the curriculum of one semester, several other topics relevant to scientific computation are not included, notably adiabatic quantum computation (AQC), and variational quantum algorithms (VQA). These materials may be added in future editions of the lecture notes.
To my knowledge, some of the materials in these lecture notes may be new and have not been presented in the literature. The sections marked by * can be skipped upon first reading without much detriment.


I would  like to thank Dong An, Yulong Dong, Di Fang, Fabian M. Faulstich, Cory Hargus, Zhen Huang, Subhayan Roy Moulik, Yu Tong, Jiasu Wang, Mathias Weiden, Jiahao Yao, Lexing Ying for useful discussions and for pointing out typos in the notes.
I would like also like to thank Nilin Abrahamsen, Di Fang, Subhayan Roy Moulik, Yu Tong for contributing some of the exercises, and Jiahao Yao for providing the cover image of the notes.
For errors / comments / suggestions / general thoughts on the lectures notes, please send me an email: \url{linlin@math.berkeley.edu}.



